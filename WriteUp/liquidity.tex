\documentclass[prodmode,acmec]{ec-acmsmall} % Aptara syntax

% Package to generate and customize Algorithm as per ACM style
\usepackage[ruled]{algorithm2e}
\renewcommand{\algorithmcfname}{ALGORITHM}
\SetAlFnt{\small}
\SetAlCapFnt{\small}
\SetAlCapNameFnt{\small}
\SetAlCapHSkip{0pt}
\IncMargin{-\parindent}

% Metadata Information
\acmVolume{9}
\acmNumber{4}
\acmArticle{39}
\acmYear{2013}
\acmMonth{6}
\usepackage[numbers]{natbib}
\usepackage{amsmath}
\usepackage{amssymb}
%\usepackage{amsthm}
\usepackage{parskip}
\usepackage{booktabs}
%new commands
\newcommand{\la}{\langle}
\newcommand{\ra}{\rangle}
\newcommand{\mX}{\mathcal{X}}
\newcommand{\bR}{\mathbf{R}}
\newcommand{\grad}{\nabla}
\newcommand{\Exp}{\mathbf{E}}
\newcommand{\Var}{\mathrm{Var}}
\newcommand{\q}{\mathbf{q}}
\newcommand{\mM}{\ensuremath{\mathcal{M}}}
\newcommand{\hmu}{\ensuremath{\hat{\mu}}}
\newcommand{\mP}{\ensuremath{\mathcal{P}}}
\newcommand{\mY}{\ensuremath{\mathcal{Y}}}
\newcommand{\mS}{\ensuremath{\mathcal{S}}}
\newcommand{\ds}{\displaystyle}
\newcommand{\mF}{\mathcal{F}}
\newcommand{\mD}{\mathcal{D}}
\newcommand{\htheta}{\hat{\theta}}
%---------------------------------------------------------------------------
\newtheorem{thm}{Theorem}[section]
\newtheorem{prop}[thm]{Proposition}
\newtheorem{lem}[thm]{Lemma}
\newtheorem{con}[thm]{Conjecture}
\newtheorem{cor}[thm]{Corollary}
\newtheorem{ex}[thm]{Example}
\newtheorem{qn}[thm]{Question}
\newcommand{\defn}{\bigbreak\noindent{\bf Definition. }}
\newcommand{\undefn}{\bigbreak}
\newcommand{\rmk}{\bigbreak\noindent{\bf Remark. }}
\newcommand{\unrmk}{\bigbreak}
%\newcommand{\proof}{\noindent{\bf Proof. }}
\newcommand{\unproof}{\hfill$\Box$\bigbreak}
\newcommand{\defeq}{\stackrel{\mathrm{def}}{=}}
\newcommand{\holds}[1]{\stackrel{?}{#1}}
\newcommand{\conv}{\mathrm{conv}}
\newcommand{\ie} {{\it i.e. }}
\newcommand{\eg} {{\it e.g. }}
\newcommand{\etal} {{\it et al. }}
\newcommand{\E}{\mathbf{E}}
\newcommand{\betavec}{\pmb{\beta}}
\newcommand{\qvec}{\mathbf{q}}
\newcommand{\lpf}{\mathbf{\psi}} %logpartition function
\newcommand{\family}{\mathcal F} 
\newcommand{\suff}{\pmb{\phi}} % suff stats
\newcommand{\muvec}{\pmb{\mu}}


% Document starts

\begin{document}
\markboth{Abernethy et al.}{Maximum Entropy Prediction Markets}
% Title portion
\title{Maximum Entropy Prediction Markets}
\author{JACOB ABERNETHY
\affil{University of Michigan, Ann Arbor}
SINDHU KUTTY
\affil{University of Michigan, Ann Arbor}
S\'{E}BASTIEN LAHAIE
\affil{Microsoft Research, New York City}
RAHUL SAMI
\affil{University of Michigan, Ann Arbor}}

\section{Cost Function with a Liquidity Parameter}
We may introduce a liquidity parameter to the cost function as $C_{\lambda}(\theta)=\lambda C\left(\frac{\theta}{\lambda}\right)$. We would like to find an expression for a lower bound on expected payoff. We would also like to find an expression for final market state for exponential utility traders with exponential family beliefs.
\subsection{lower bound on expected payoff}
The prediction market is set up with cost function $C_{\lambda}(\theta)$  and payoff for a unit share vector as $\phi(x)$. To find a lower bound on expected payoff, we first note that
\begin{eqnarray*}
\E_{x\sim p(x;\beta)}[C_{\lambda}(\theta)-\theta\phi(x)]&=&C_{\lambda}(\theta)-\theta\nabla T(\beta)\\
&=&\lambda T(\theta/\lambda)-\theta\nabla T(\beta)\\
&=&\lambda [T(\theta/\lambda)-\theta/\lambda\nabla T(\beta)-T(\beta)+\beta\nabla T(\beta)]+\lambda T(\beta)-\theta\lambda\nabla T(\beta)\\
&=&\lambda D_{T}(\theta/\lambda,\beta)+\lambda [T(\beta)-\theta\nabla T(\beta)]
\end{eqnarray*}� 
where $D_{T}(x_{1},x_{2}) = T(x_{1})-T(x_{2})-\nabla T(x_{1}-x_{2})$ is the Bregman Divergence between $x_{1}$ and $x_{2}$ based on $T$.

Thus, the expected payoff  for a trader with exponential family belief parameter $\beta$ who moves the market state from $\theta$ to $\theta'$ can be written as
\begin{eqnarray*}
\E_{x\sim p(x;\beta)}\left[C_{\lambda}(\theta)-\theta\phi(x)-(C_{\lambda}(\theta')-\theta'\phi(x))\right]=\lambda[D_{T}(\theta/\lambda,\beta)-D_{T}(\theta'/\lambda,\beta)]
\end{eqnarray*}�
If $\theta'=a\beta+(1-a)\theta$ for some $a\in[0,1]$ we can write
\begin{eqnarray*}
D_{T}(\theta'/\lambda,\beta)&\leq& a D_{T}(\beta/\lambda,\beta)+(1-a) D_{T}(\theta/\lambda,\beta)\\
D_{T}(\theta/\lambda,\beta)-D(\theta'/\lambda,\beta)&\geq& a D_{T}(\theta/\lambda,\beta)-a D_{T}(\beta/\lambda,\beta)
\end{eqnarray*}�
%Note that 
%\begin{eqnarray*}
%\nabla_{\theta} C_{\lambda} (\theta)&=&\nabla_{\theta}\lambda C\left(\frac{\theta}{\lambda}\right)\\
%&=&\nabla_{\theta/\lambda}C\left(\frac{\theta}{\lambda}\right)\\
%&=&\E_{\theta/\lambda}[\phi(x)]
%\end{eqnarray*}�
\subsection{exponential utility traders with exponential family beliefs}
As before the prediction market is set up with cost function $C_{\lambda}(\theta)$  and payoff for a unit share vector as $\phi(x)$. Thus, the  payoff of a trader who moves the market state from $\theta$ to $\theta'$ is given by $C_{\lambda}(\theta)-C_{\lambda}(\theta')+\delta\phi(x)$ where $\delta=\theta'-\theta$ and the expected utility of the payoff for a trader with exponential belief parameter $\beta$ is given by
\begin{eqnarray*}
&&\E_{x\sim p(x;\beta)}U[C_{\lambda}(\theta)-C_{\lambda}(\theta')+\delta\phi(x)]\\
&=&U[C_{\lambda}(\theta)-C_{\lambda}(\theta+\delta)+\frac{1}{a}(T(\beta)-T(\beta-a\delta))]\\
&=&U[\lambda C(\theta/\lambda)-\lambda C\left(\frac{\theta+\delta}{\lambda}\right)+\frac{1}{a}(T(\beta)-T(\beta-a\delta))]\\
&=&U[\lambda T(\theta/\lambda)-\lambda T\left(\frac{\theta+\delta}{\lambda}\right)+\frac{1}{a}(T(\beta)-T(\beta-a\delta))]
\end{eqnarray*}�
Here $T$ is the log partition function; for the exponential family prediction market $C=T$. To maximize the utility, we set the gradient of the argument with respect to $\delta$ to 0. 
$$\nabla T\left(\frac{\theta+\delta}{\lambda}\right) = \nabla T(\beta-a\delta)$$
So we have for an optimal choice of $\delta$, 
\begin{eqnarray*}
\frac{\theta+\delta}{\lambda} &=& \beta-a\delta\\
\implies \delta&=&\frac{\beta\lambda-\theta}{1+a\lambda}\\
\implies \theta+\delta&=&\lambda\frac{\beta+a\delta}{1+a\lambda}
\end{eqnarray*}




\subsection{Introducing Liquidity (S\'{e}bastien Remarks)}

We have been assuming that the cost function is the same as the log partition function, but this assumes a liquidity setting of 1. Consider the parametrized  cost $C_{\lambda}(\theta)=\frac{1}{\lambda} C\left(\lambda\theta\right)$. Here $\lambda$ is construed as the \emph{inverse liquidity}, or price responsiveness. Higher $\lambda$ means fewer shares need to be bought to reach the same prices.

(Discussion, you can skip this paragraph.) This is because for higher $\lambda$ we are making the cost function ``more convex''. Specifically, for $\lambda_1 > \lambda_2$ it's the case that $C_{\lambda_1}$ is an increasing convex transformation of $C(\lambda_2)$. This holds because the ``coefficient of risk aversion'' $A(\theta) = C''(\theta)/C'(\theta)$ increases with $\lambda$ at every point. I've written it assuming single-dimensional $\theta$ but there are more complicated looking generalizations for vector $\theta$. In fact, this coefficient might be the ``right'' notion of liquidity at a point, not the second derivative or largest eigenvalue of the Hessian. We should discuss.

\newcommand{\ttheta}{\tilde{\theta}}
As we're dealing with LMSR I write $T$ rather than $C$ now where $T$ is the log partition function. The cost function is the liquidity-adjusted version $T_{\lambda}$. Let $\hmu$ be the agent's mean belief and let $\htheta = \grad T^{-1}(\hmu)$ be the corresponding natural parameter. A risk-neutral agent moves the share vector so that the prices match its mean parameter. Therefore, define the \emph{target shares} as $\ttheta = \grad T^{-1}_{\lambda}(\hmu)$. So $\htheta$ and $\ttheta$ are related by
%
$$
\grad T(\htheta) = \grad T_{\lambda}(\ttheta).
$$
%
It's easy to check that $\grad T_{\lambda}(\ttheta) = \grad T(\lambda \ttheta)$ so we have
%
\begin{equation} \label{target}
\ttheta = \htheta / \lambda.
\end{equation}
%
Now using the liquidity-adjusted cost function, Equation~(17) from Theorem~6.2 in the write-up becomes the following, where $\theta$ is the current market state:
%
\begin{equation} \label{update}
\frac{\lambda}{\lambda+a} \ttheta + \frac{a}{\lambda+a} \theta.
\end{equation}
%
As $\lambda$ grows large, the agent moves the market state closer to the \emph{target shares}, rather than its true natural parameter. Note that the target shares themselves depend on $\lambda$, because of~(\ref{target}). This make sense because higher price responsiveness means the agent needs to buy fewer shares to make the market prices match its own expectations. In that sense, it may be conceptually cleaner to write~(\ref{update}) as
%
\begin{equation}
\frac{\lambda}{\lambda+a} \grad T^{-1}_{\lambda}(\hmu) + \frac{a}{\lambda+a} \grad T^{-1}_{\lambda}(\mu).
\end{equation}
%
where $\mu$ are the current market prices.







\end{document}